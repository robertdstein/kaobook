\setchapterstyle{kao}
\setchapterpreamble[u]{\margintoc}
\chapter{Introduction}
\labch{intro}
\begin{fquote}[Douglas Adams][The Restaurant at the End of the Universe][1980] In the beginning the Universe was created. This has made a lot of people very angry and been widely regarded as a bad move. 
\end{fquote}

Multi-messenger astronomy is the youngest branch of humanity's oldest scientific discipline. Understanding the stars has long been of central importance to society, and indeed guided exploration and scientific development of the human race for many millennia. While the centrality of astronomy as a tool for navigation has long past, it continues to be a key driver of the most human of pursuits, the expansion of our understanding of the universe in which we live. 

\begin{marginfigure}
	\centering \includegraphics{intro/V-Hess-web_2.jpg}
	\caption{Victor Hess with his famous balloon, 1912.}
\end{marginfigure}

The broad field of \emph{astroparticle physics}, particle physics with astrophysical accelerators, has been a fruitful source of mysteries in recent years. Astroparticle physics offers us the opportunity to overcome physical limitations of engineering in our quest for knowledge. While terrestrial accelerators such as the Large Hadron Collider at CERN may be capable of accelerating particles to $\sim$10 TeV, studies of cosmic rays in the past century have revealed that the universe can accelerate particles to $\sim$10 EeV, one billion times higher. Cosmic messengers can also test physics across distances of $\sim$10$^{23}$ km, rather than the $\sim$10$^{3}$ km of the Earth, and across timescales of billions of years rather than the typical decade-length horizon of research funding agencies. If we wish to study physics at these scales, competing with nature is not a viable option. Instead, we must learn to understand the dataset that nature provides for us.

The existence of these high-energy charged particles \emph{cosmic rays} was first demonstrated by Victor Hess in 1912, but their origin remains unknown over a century later. In that century, the ghostly neutrino particle was first proposed by Pauli in 1930, its centrality in solar fusion was first posited by Bethe in 1939, its existence was confirmed by Cowan and Reines in 1956, the solar neutrino flux was then measured in 1964 by the Homestake with an unexpectedly-low rate (the so-called \emph{solar neutrino problem}), which then led in 2001 to the first discovery of particle physics beyond the \emph{Standard Model}, namely \emph{neutrino oscillations}. 

Experiments studying these solar neutrinos also accidentally provided us with the first case of observational astronomy with multiple 'messengers' (the observation of nearby supernova \emph{SN1987a} with both neutrinos and photons), followed by the discovery of extra-terrestrial high-energy \emph{astrophysical neutrinos} with the IceCube detector in 2013 produced by the very same cosmic accelerators responsible for cosmic rays. A scramble to find the origin of these neutrinos led to the birth of a new field, \emph{neutrino astronomy}. 

The long-awaited discovery of Gravitational Waves by LIGO in 2015 was shortly thereafter followed by the discovery of a Binary Neutrino Star merger, \emph{GW170817}, detected simultaneously with both photons and gravitational waves, kick-starting \emph{gravitational-wave astronomy}. One month later, the observation of a high-energy neutrino IC170922A from the direction of a flaring galactic nucleus led to the identification of the first likely source of high-energy neutrinos, \emph{TXS 0506+056}. Thirty years after \emph{SN1987a}, the year 2017 truly marked the dawn of an era of \emph{multi-messenger astronomy}. 

This thesis presents research probing the intersect of these new branches of astronomy with multiple messengers, incorporating searches for sources of neutrinos and gravitational waves using photons, and searches for photon sources using neutrinos. At its core, it seeks to understand what can be learned through combining knowledge from these new branches of astronomy with that of the oldest, namely astronomy with optical telescopes. The latter field is undergoing a revolution of its own, at the brink of transition to an algorithm-driven era dominated by enormous data volumes. Optical astronomy is moving from object-centric to population-centric science, with scales at which detailed study of individual objects is becoming infeasible. In all three fields, a focus on rapid automated responses seeks to remove human-dependent latency in observational decisions, so-called \emph{realtime astronomy}. This drive was central in the identification of both GW170817 and TXS 0506+056, and forms a central part of this work.

This thesis begins with an introduction to ...
...



