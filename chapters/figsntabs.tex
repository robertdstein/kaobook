\setchapterimage[6.5cm]{seaside}
\setchapterpreamble[u]{\margintoc}
\chapter[Figures and Tables]{Figures and Tables\footnotemark[0]}

\footnotetext{The credits for the image above the chapter title go to:
	Bushra Feroz --- Own work, CC~BY-SA~4.0, 
	\url{https://commons.wikimedia.org/w/index.php?curid=68724647}}

\section{Normal Figures and Tables}

Figures and tables can be inserted just like in any standard 
\LaTeX\xspace document. The \Package{graphicx} package is already loaded 
and configured in such a way that the figure width is equal to the 
textwidth and the height is adjusted in order to maintaini the original 
aspect ratio. As you may have imagined, the captions will be 
positioned\ldots well, in the margins. This is achieved with the help of 
the \Package{floatrow} package.

Here is a picture of Mona Lisa (\reffig{normalmonalisa}), as an example. 
The captions are formatted as the margin- and the side-notes; If you 
want to change something about captions you can use the command 
\Command{captsetup} from the \Package{caption} package. Remember that if 
you want to reference a figure, the label must come \emph{after} the 
caption!

\begin{figure}[hb]
	\includegraphics[width=0.45\textwidth]{monalisa}
	\caption[Mona Lisa, again]{It's Mona Lisa again. \blindtext}
	\labfig{normalmonalisa}
\end{figure}

While the format of the caption is managed by \Package{caption}, its 
position is handled by the \Package{floatrow} package. Achieving this 
result has been quite hard, but now I am pretty satisfied. In two-side 
mode, the captions are printed in the correct margin.

Tables can be inserted just as easily as figures, as exemplified by the 
following code:

\begin{lstlisting}
\begin{table}
\begin{tabular}{ c c c c }
	\toprule
	col1 & col2 & col3 & col 4 \\
	\midrule
	\multirow{3}{4em}{Multiple row} & cell2 & cell3 & cell4\\ &
	cell5 & cell6 & cell7 \\ &
	cell8 & cell9 & cell10 \\
	\multirow{3}{4em}{Multiple row} & cell2 & cell3 & cell4 \\ &
	cell5 & cell6 & cell7 \\ &
	cell8 & cell9 & cell10 \\
	\bottomrule
\end{tabular}
\end{table}
\end{lstlisting}

which results in the useless \vreftab{useless}.

\begin{table}[h]
\caption[A useless table]{A useless table.}
\labtab{useless}
\begin{tabular}{ c c c c }
	\toprule
	col1 & col2 & col3 & col 4 \\
	\midrule
	\multirow{3}{4em}{Multiple row} & cell2 & cell3 & cell4\\ &
	cell5 & cell6 & cell7 \\ &
	cell8 & cell9 & cell10 \\
	\multirow{3}{4em}{Multiple row} & cell2 & cell3 & cell4 \\ &
	cell5 & cell6 & cell7 \\ &
	cell8 & cell9 & cell10 \\
	\bottomrule
\end{tabular}
\end{table}

I don't have much else to say, so I will just insert some blind text. 
\blindtext

\section{Margin Figures and Tables}

Marginfigures can be inserted with the environment 
\Environment{marginfigure}. In this case, the whole picture is confined 
to the margin and the caption is below it. \reffig{marginmonalisa} is 
obtained with something like this:

\begin{lstlisting}
\begin{marginfigure}
	\includegraphics{monalisa}
	\caption[The Mona Lisa]{The Mona Lisa.}
	\labfig{marginmonalisa}
\end{marginfigure}
\end{lstlisting}

There is also the \Environment{margintable} environment, of which 
\reftab{anotheruseless} is an example. Notice how you can place the 
caption above the table by just placing the \Command{caption} command 
before beginning the \Environment{tabular} environment. Usually, figure 
captions are below, while table captions are above. This rule is also 
respected for normal figures and tables: the captions are always on the 
side, but for figure they are aligned to the bottom, while for tables to 
the top.

\begin{margintable}
\caption[Another useless table]{Another useless table.}
\labtab{anotheruseless}
\raggedright
\begin{tabular}{ c c c c }
	\hline
	col1 & col2 & col3 \\
	\hline
	\multirow{3}{4em}{Multiple row} & cell2 & cell3 \\ & cell5 & cell6 
	\\ & cell8 & cell9 \\ \hline
\end{tabular}
\end{margintable}

Marginfigures and tables can be positioned with an optional offset 
command, like so:

\begin{lstlisting}
\begin{marginfigure}[offset]
	\includegraphics{images/seaside}
\end{marginfigure}
\end{lstlisting}

Offset ca be either a measure or a multiple of \Command{baselineskip}, 
much like with \Command{sidenote}, \Command{marginnote} and 
\Command{margintoc}.\todo{Improve this part.} If you are wondering how I 
inserted this orange bubble, have a look at the \Package{todo} package.

\section{Wide Figures and Tables}

\begin{figure*}[h!]
	\includegraphics{seaside}
	\caption[A wide seaside]{A wide seaside, and a wide caption.
		Credits: By Bushra Feroz --- Own work, CC BY-SA 4.0, 
		\url{https://commons.wikimedia.org/w/index.php?curid=68724647}}
\end{figure*}

With the environments \Environment{figure*} and \Environment{table*} you 
can insert figures which span the whole page width. The caption will be 
positioned below or above, according to taste.

You may have noticed the full width image at the very beginning of this 
chapter: that, however, is set up in an entirely different way, which 
you'll read about in \vrefch{layout}. Now it is time to tackle 
hyperreferences.
